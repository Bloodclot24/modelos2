\documentclass{article}

\usepackage[spanish]{babel}
\selectlanguage{spanish}
\usepackage[utf8]{inputenc}

\title{Trabajo Práctico 1\\Teoría de Colas}
\author{}
\date{\today}

%%% Agregar integrantes del grupo %%%

\begin{document}
\maketitle

%%%% Poner el enunciado aca.. y un indice %%%%
\paragraph{Ejercicio 1}
    El modelo elegido para este sistema es P/P/1. Las distribuciones Poisson de los arribos y las salidas concuerdan con las de este modelo, y además se explicita que no hay impaciencia. Se asumen también las siguientes hipótesis:
    \begin{itemize}
        \item El tipo de arribo responde a distribución Poisson; es dato.
        \item El tipo de proceso de servicio responde a distribución Poisson; también es dato, ya que el tiempo de servicio tiene distribución exponencial. 
        \item Un único canal de atención; es dato.
        \item Sistema de capacidad infinita.
        \item Disciplina de atención FIFO.
        \item Población infinita.
        \item Cola única.
        \item Población sin impaciencia; es dato.
        \item Sistema en régimen permanente.
    \end{itemize}
    
    Los datos proporcionados son la velocidad de ingreso $ \lambda = 24/h $ y el tiempo promedio de atención $ t_s = 2 min $, a partir del cual podemos calcular la tasa de atención $ \mu = \frac{1}{t_s} = \frac{1}{2min} \; \frac{60min}{h} = \frac{30}{h} $.

    \paragraph{1.a.} El número promedio de clientes en la sección de sastrería equivale a la cantidad de clientes en nuestro sistema, $ L $. De la combinación de la tabla de estados y la ecuación de estado para un sistema P/P/1 se obtiene la siguiente expresión para $ L $:

    $$ L = \frac{\lambda}{\mu - \lambda} = \frac{24 h^{-1}}{30 h^{-1} - 24 h^{-1}} = \frac{24}{6} = 4 $$

    \paragraph{1.b.} Equivale al tiempo promedio de permanencia en el sistema de un cliente. Puede obtenerse como el tiempo de permanencia en la cola más el tiempo de atención del cliente:
    
    $$ w = w_c + t_s = \frac{L_c}{\bar\lambda} + t_s = \frac{\lambda^2}{\lambda\mu(\mu-\lambda)} + t_s = \frac{\lambda}{\mu(\mu-\lambda)} + t_s $$
    $$ w = \frac{\lambda}{\mu(\mu-\lambda)} + t_s = \frac{24 h^{-1}}{30 h^{-1}(30 h^{-1} - 24 h^{-1})} + 2 min = \frac{24 h^{-1}}{30 h^{-1} \; 6 h^{-1}} + 2 min $$
    $$ w = \frac{2}{15} h \; \frac{60min}{h} + 2 min = 8 min + 2 min $$
    $$ w = 10 min $$

    \paragraph{1.c.} Corresponde a la probabilidad de que no haya clientes en el sistema, siendo para P/P/1:
    
    $$ p(0) = 1 - \rho = 1 - \frac{\lambda}{\mu} = 1 - \frac{24h^{-1}}{30h^{-1}} = 1 - \frac{4}{5} = \frac{1}{5} = 0.2 $$
    
    \paragraph{1.d.} Lo que se pide es el número promedio de clientes en la cola:
    
    $$ L_c = \frac{\lambda^2}{\mu(\mu - \lambda)} = \frac{24^2h^{-2}}{30h^{-1}(30h^{-1}-24h^{-1})} = \frac{576h^{-1}}{30 \cdot 6h^{-1}} = \frac{576}{180} = \frac{16}{5} = 3.2 $$

\paragraph{Ejercicio 2}
    El sistema en cuestión es similar al anterior, puede modelarse como P/P/1 con las mismas hipótesis. Los datos son $ \lambda = 4h^{-1} $, $ t_s = 6min \Rightarrow \mu = \frac{1}{6min} \; \frac{60min}{h} = 10h^{-1} $.
    
    \paragraph{2.a.} Corresponde a la probabilidad de que no haya clientes en el sistema, siendo para P/P/1:
    
    $$ p(0) = 1 - \rho = 1 - \frac{\lambda}{\mu} = 1 - \frac{4h^{-1}}{10h^{-1}} = 1 - \frac{2}{5} = \frac{3}{5} = 0.6 $$
    
    \paragraph{2.b.} 
    $$ p(3) = \rho^{3}p(0) = \left( \frac{\lambda}{\mu} \right)^{3} \frac{3}{5} = \left( \frac{2}{5} \right)^{3} \frac{3}{5} = \frac{8\cdot3}{125\cdot5} = \frac{24}{625} = 0.0384 $$
    
    \paragraph{2.c.} La probabilidad de encontrar al menos un cliente en el taller es la probabilidad de que haya entre uno e infinitos clientes: $ \sum _{n=1}^{\infty} {p(n)} $. Teniendo en cuenta que $ \sum _{n=0}^{\infty} {p(n)} = p(0) + \sum _{n=1}^{\infty} {p(n)} = 1 $ podemos concluir:
     
     $$ \sum _{n=1}^{\infty} {p(n)} = 1 - p(0) = 1 - (1 - \rho) = \rho = \frac{\lambda}{\mu} = \frac{2}{5} = 0.4 $$
    
    \paragraph{2.d.}
    $$ L = \frac{\lambda}{\mu - \lambda} = \frac{4h^{-1}}{10h^{-1} - 4h^{-1}} = \frac{4h^{-1}}{6h^{-1}} = \frac{2}{3} = 0.\overline{6} $$
    
    \paragraph{2.e.} 
    $$ w = w_c + t_s = \frac{\lambda}{\mu(\mu - \lambda)} + 6min = \frac{4h^{-1}}{10h^{-1}(10h^{-1} - 4h^{-1})} + 6min $$
    $$ w = \frac{4}{10\cdot6h^{-1}} + 6min = \frac{h}{15}\frac{60min}{h} + 6min = 4min + 6 min = 10min $$
    
    \paragraph{2.f.} Equivale a la longitud de la cola:
    
    $$ L_c = \frac{\lambda^2}{\mu(\mu - \lambda)} = \frac{4^2 h^{-2}}{10h^{-1}(10h^{-1} - 4h^{-1})} = \frac{16h^{-2}}{10\cdot 6h^{-2}} = \frac{4}{15} = 0.2\overline{6} $$
    
    \paragraph{2.g.}
    $$ w_c = \frac{L_c}{\lambda} = \frac{4}{15} \frac{h}{4} = \frac{h}{15} \frac{60min}{h} = 4 min $$
    
\paragraph{Ejercicio 3}
    Este problema también corresponde a un modelo P/P/1, ya que los tiempos entre arribos y de servicio con distribución exponencial corresponden a arribos y salidas con distribución Poisson. Se toman las mismas hipótesis que en los ejercicios anteriores. Los datos son: $ t_a = 8min $, $ t_s = 2min $.

    \paragraph{3.a.} La probabilidad de que un cliente tenga que esperar es la probabilidad de que el canal esté ocupado, es decir que haya más de un cliente en el sistema:
    
    $$ \sum _{n=1}^{\infty} p(n) = 1 - p(0) = 1 - (1 - \rho) = \rho = \frac{\lambda}{\mu} = \frac{t_s}{t_a} = \frac{2min}{8min} = \frac{1}{4} = 0.25 $$    

    \paragraph{3.b.}
    $$ L_c = \frac{\lambda^2}{\mu(\mu - \lambda)} = \frac{\frac{1}{8^{2}} min^{-2}}{\frac{1}{2} min^{-1} \left( \frac{1}{2} - \frac{1}{8} \right) min^{-1}} $$
    $$ L_c = \frac{1}{64} :   \left( \frac{1}{2} \cdot \frac{3}{8} \right) = \frac{1}{64} \cdot \frac{16}{3} = \frac{1}{12} = 0.08\overline{3} $$

    \paragraph{3.c.} Para averiguar la velocidad promedio de arribos $ \overline \lambda $ que resultaría en tiempos de espera en la cola mayores a 4 minutos, se debe despejar dicha variable de la fórmula de espera promedio en la cola. En el modelo P/P/1 se cumple que $ \lambda = \overline \lambda $, por lo tanto:
    
    $$ w_c = \frac{\lambda}{\mu(\mu - \lambda)} > 4 min $$

    Aplicando la función monótonamente decreciente $ f(x) = x^{-1} $ resulta:

    $$ \frac{\mu(\mu - \lambda)}{\lambda} < \frac{1}{4min} \Rightarrow \mu(\mu - \lambda) < \frac{\lambda}{4min} $$
    $$ \mu^2 - \mu \lambda < \frac{\lambda}{4min} \Rightarrow \mu^2 < \lambda \left( \frac{1}{4min} + \mu \right) $$
    $$ \frac{\mu^2}{\frac{1}{4min} + \mu} < \lambda \Rightarrow \frac{\left( \frac{1}{2min} \right)^2}{\frac{1}{4min} + \frac{1}{2min}} < \lambda $$
    $$ \frac{1}{4min} \cdot \frac{4}{3} < \lambda $$
    $$ \frac{1}{3min} < \lambda $$
    
    La velocidad promedio de arribos para que la espera supere los 4 minutos debería ser al menos de un cliente cada 3 minutos.


\paragraph{Ejercicio 4}
    El sistema en cuestión es similar al del ejercicio 2, exceptuando que ahora el sistema tiene una capacidad limitada, 
el mismo, puede modelarse como P/P/1/3. Los datos son $ \lambda = 4h^{-1} $, $ t_s = 6min \Rightarrow \mu = \frac{1}{6min} \; \frac{60min}{h} = 10h^{-1} $.
    
    \paragraph{4.a.} Corresponde a la probabilidad de que no haya clientes en el sistema, siendo para P/P/1/3:
     La siguiente tabla describe al sistema:
    \begin{center}
    \begin{tabular}{|| c | c | c | c | c | c | c | c ||}
    \hline 
     n & P(n) & $\lambda_n$ & $\mu_n$ & $L_n$& $L_cn$ & $H_n$ & $R_n$ \\ \hline \hline
     0 & P(0) & $\lambda$   & 0       & 0    & 0      & 0     & 0	\\ \hline
     1 & P(1) & $\lambda$   & $\mu$   & 1    & 0      & 1     & 0	\\ \hline
     2 & P(2) & $\lambda$   & $\mu$  & 2    & 1      & 1     & 0	\\ \hline
     3 & P(3) & 0 & $\mu$  & 3    & 2      & 1     & $\lambda$ \\ \hline
     
    \end{tabular}
    \end{center}
    
    $$\rho = \frac{\lambda}{\mu} = \frac{4}{10}=0.4 $$

  Utilizando que: 
    $$ p(n) = \rho^{n} p(n)$$
    $$ \sum _{n=1}^{\infty} {p(n)} = 1$$
  Entonces:

    $$ p(1) = \rho^{1} p(0) = 0.4 p(0)$$
    $$ p(2) = \rho^{2} p(0) = 0.16 p(0)$$
    $$ p(3) = \rho^{3} p(0) = 0.064 p(0)$$
    
    $$ \sum _{n=1}^{\infty} {p(n)} = p(0) + p(1) + p(2) + p(3) = 1$$
    $$p(0) + p(1) + p(2) + p(3) = 1$$
    $$p(0) + 0.4 p(0) + 0.16 p(0) + 0.064 p(0) = 1$$
    $$1.624 p(0) = 1$$
    $$p(0) = \frac{1}{1.625}$$

  Y por lo tanto, la probabilidad de que el taller este vac\'io es igual a:
    $$p(0) = 0.616$$
    
    \paragraph{4.b.}
    Utilizando la probabilidad calculada en el punto anterior, se obtiene que la probabilidad de que haya 3 clientes en el taller es:

    $$ p(3) = \rho^{3}p(0) =  0.064 * 0.616 = 0.0394 $$
    
    \paragraph{4.c.} La probabilidad de encontrar al menos un cliente en el taller es la probabilidad de que haya entre uno y 3 clientes: $ \sum _{n=1}^{\infty} {p(n)} $. Teniendo en cuenta que $ \sum _{n=0}^{\infty} {p(n)} = p(0) + \sum _{n=1}^{\infty} {p(n)} = 1 $ podemos concluir:
     
     $$ \sum _{n=1}^{3} {p(n)} = p(1) + p(2) + p(3) = 0.4 p(0) + 0.16 p(0) + 0.064 p(0) = 0.4 * 0.616 + 0.16 * 0.616 + 0.064 * 0.616 = 0.384 $$
    
    \paragraph{4.d.}
    $$ L = \sum _{n=0}^{3} {p(n)} = 0 p(0) + 1 p(1) + 2 p(2) + 3 p(3) = 0 +  0.246 + 0.197 + 0.118 = 0.561$$
    
    \paragraph{4.e.} 
    $$ w = w_c + t_s = \frac{L_c}{\overline{\lambda}} + 6min $$
    Calculando:
      $$\overline{\lambda} = 4 hr^-1 . p(0) + 4 hr^-1 . p(1) + 4 hr^-1 . p(2) + 0 hr^-1 . p(3) = 4 * 0.616 + 4 * 0.246 + 4 * 0.0986 + 0 = 3.842$$
     $$ L_c = 0 p(0) + 0 p(1) + 1 p(2) + 2 p(3) = p(2) + 2 p(3) = 0.0986 + 2 * 0.0394 = 0.887$$
    $$ w_c = \frac{L_c}{\overline{\lambda}} = \frac{0.877}{3.842} = 0.23$$
    Por lo tanto
    $$ w = 0.23h + 6 min = 13.8 min + 6 min = 19.8 min $$
    
    \paragraph{4.f.} Equivale a la longitud de la cola, utilizando lo calculado en el punto anterior:
    
    $$ L_c = 0.887$$
    
    \paragraph{4.g.} Utilizando lo calculado en el punto 4.e
    $$ w_c = 0.23 $$
    \paragraph{4.h.} Equivale a calcular:

    $$\overline{R} = \lambda - \overline{\lambda} = 4 - 3.842 = 0.158$$

\paragraph{Ejercicio 5}
  Este sistema se puede modelar utilizando P/P/1 (N') es decir, es un sistema P/P/1 con poblaci\'on finita, en este caso, las 4 cortadoras de c\'esped.
  
  Los datos brindados por el enunciado son los siguientes: $t_s = 7 dias\Rightarrow \mu = \frac{1}{7dias} = 0.143dias^{-1} $  $t_r = 15 dias\Rightarrow\lambda = \frac{1}{15dias} = 0.066dias^{-1}$
  
  Para este tipo de sistemas, debe tenerse en cuenta que:
  $$\lambda_n = \lambda_r (N' - n)$$
    La siguiente tabla describe al sistema:
    \begin{center}
    \begin{tabular}{|| c | c | c | c | c | c | c ||}
    \hline 
     n & P(n) & $\lambda_n$ & $\mu_n$ & $L_n$& $L_cn$ & $H_n$  \\ \hline \hline
     0 & P(0) & $4\lambda_r$   & 0       & 0    & 0      & 0  \\ \hline
     1 & P(1) & $3\lambda_r$   & $\mu$   & 1    & 0      & 1  \\ \hline
     2 & P(2) & $2\lambda_r$   & $\mu$  & 2    & 1      & 1    	\\ \hline
     3 & P(3) & $\lambda_r$ & $\mu$  & 3    & 2      & 1   \\ \hline
     4 & P(4) & $0$       & $\mu$  & 4    & 3      & 1       \\ \hline  
    \end{tabular}
    \end{center}
  
\paragraph{5.a.} La cantidad de m\'aquinas funcionando puede obtenerse como:
  $$J = N' - \overline{L}$$
  $$J = 4 - \overline{L} $$

  Entonces, se calcula $\overline{L}$:
  $$\overline{L} = 0 p(0) + 1 p(1) + 2 p(2) + 3 p(3) + 4 p(4) $$

  Utilizando que: $\lambda_n = \lambda_r (N' - n)$:
    $$\lambda_0 = \lambda_r (4 - 0)$$
    $$\lambda_0 = 0.066 dias^-1 * 4 = 0.264 $$
    $$\lambda_1 = \lambda_r (4 - 1)$$
    $$\lambda_1 = 0.066 dias^-1 * 3 = 0.198 $$
    $$\lambda_2 = \lambda_r (4 - 2)$$
    $$\lambda_2 = 0.066 dias^-1 * 2 = 0.132 $$
    $$\lambda_3 = \lambda_r (4 - 3)$$
    $$\lambda_3 = 0.066 dias^-1 * 1 = 0.066 $$
    
    $$\rho = \frac{\lambda}{\mu} = \frac{0.066}{0.143}=0.462 $$ 

  Entonces, se calculan las probabilidades considerando que:
      $$ p(n+1) = \frac{\lambda_n \cdot p(n)}{\mu_n}$$
    $$ \sum _{n=1}^{\infty} {p(n)} = 1$$
  Entonces:

    $$ p(1) = \frac{\lambda_0 \cdot p(0)}{\mu_1} = \frac{0.246 \cdot p(0)}{0.143} = 1.720 \cdot p(0)  $$
    $$ p(2) = \frac{\lambda_1 \cdot p(1)}{\mu_2} = \frac{0.198 \cdot p(0)}{0.143} = 1.385 \cdot p(0) $$
    $$ p(3) = \frac{\lambda_2 \cdot p(2)}{\mu_3} = \frac{0.132 \cdot p(0)}{0.143} = 0.923 \cdot p(0)$$
    $$ p(4) = \frac{\lambda_3 \cdot p(3)}{\mu_4} = \frac{0.066 \cdot p(0)}{0.143} = 0.462 \cdot p(0)$$

    $$ \sum _{n=1}^{\infty} {p(n)} = p(0) + p(1) + p(2) + p(3) + p(4) = 1$$
    $$p(0) + p(1) + p(2) + p(3) = 1$$
    $$p(0) + 1.720 p(0) + 1.385 p(0) + 0.923 p(0) + 0.462 p(0) = 1$$
    $$5.49 p(0) = 1$$
    $$p(0) = \frac{1}{5.49}$$
    $$p(0) = 0.182$$

  $$p(1) = 0.313$$
  $$p(2) = 0.252$$
  $$p(3) = 0.168$$
  $$p(4) = 0.084$$

Por lo tanto:
$$\overline{L} = 0 * 0.182 + 1 * 0.313 + 2 * 0.252 + 3 * 0.168 + 4 * 0.084 = 1.657 $$

$$J = 4 - 1.657 = 2.343 $$
  
\paragraph{5.b.} Esto se puede calcular como:
  $$T_{inactivo} = 1 - H$$
 Entonces calculo:
  $$H = p(1) + p(2) + p(3) + p(4) = 0.313 + 0.252 + 0,168 + 0.084 = 0.817$$
Por lo tanto:
  $$T_{inactivo} = 1 - 0.74 = 0.26$$

\paragraph{5.c.} El tiempo de funcionamiento en promedio de una m\'aquina se puede calcular como:

$$T_{funcionando} = T_r$$
Entonces:
 $$T_{funcionando} = 15 dias$$
    
\paragraph{5.d.} Considerando que la cantidad promedio de  m\'aquinas funcionando es 2.343 y que el beneficio que obtengo es de 50\$ por cada dia que las m\'aquinas funcionan;
 En un mes se observa un beneficio de: $Beneficio = 2.343 * 50\$ * 24 dias =  \$2811.6 $
Si tengo 2 operarios, planteo un nuevo modelo con 2 canales de atenci\'on:
La siguiente tabla describe al sistema:
    \begin{center}
    \begin{tabular}{|| c | c | c | c | c | c | c ||}
    \hline 
     n & P(n) & $\lambda_n$ & $\mu_n$ & $L_n$& $L_cn$ & $H_n$  \\ \hline \hline
     0 & P(0) & $4\lambda_r$   & 0       & 0    & 0      & 0  \\ \hline
     1 & P(1) & $3\lambda_r$   & $\mu$   & 1    & 0      & 1  \\ \hline
     2 & P(2) & $2\lambda_r$   & 2$\mu$  & 2    & 0      & 2    	\\ \hline
     3 & P(3) & $\lambda_r$ & 2$\mu$  & 3    & 1      & 2   \\ \hline
     4 & P(4) & $0$       & 2$\mu$  & 4    & 2      & 2       \\ \hline  
    \end{tabular}
    \end{center}
  
$$J = N' - \overline{L}$$
  $$J = 4 - \overline{L} $$

  Entonces, se calcula $\overline{L}$:
  $$\overline{L} = 0 p(0) + 1 p(1) + 2 p(2) + 3 p(3) + 4 p(4) $$

  Utilizando que: $\lambda_n = \lambda_r (N' - n)$:
    $$\lambda_0 = \lambda_r (4 - 0)$$
    $$\lambda_0 = 0.066 dias^-1 * 4 = 0.264 $$
    $$\lambda_1 = \lambda_r (4 - 1)$$
    $$\lambda_1 = 0.066 dias^-1 * 3 = 0.198 $$
    $$\lambda_2 = \lambda_r (4 - 2)$$
    $$\lambda_2 = 0.066 dias^-1 * 2 = 0.132 $$
    $$\lambda_3 = \lambda_r (4 - 3)$$
    $$\lambda_3 = 0.066 dias^-1 * 1 = 0.066 $$
    
  Entonces, se calculan las probabilidades considerando que:
      $$ p(n+1) = \frac{\lambda_n \cdot p(n)}{\mu_n}$$
    $$ \sum _{n=1}^{\infty} {p(n)} = 1$$
  Entonces:

    $$ p(1) = \frac{\lambda_0 \cdot p(0)}{\mu_1} = \frac{0.246 \cdot p(0)}{0.143} = 1.720 \cdot p(0)  $$
    $$ p(2) = \frac{\lambda_1 \cdot p(1)}{\mu_2} = \frac{0.198 \cdot p(0)}{2 \cdot 0.143} = 0.692 \cdot p(0) $$
    $$ p(3) = \frac{\lambda_2 \cdot p(2)}{\mu_3} = \frac{0.132 \cdot p(0)}{2 \cdot 0.143} = 0.462 \cdot p(0)$$
    $$ p(4) = \frac{\lambda_3 \cdot p(3)}{\mu_4} = \frac{0.066 \cdot p(0)}{2 \cdot 0.143} = 0.231 \cdot p(0)$$

    $$ \sum _{n=1}^{\infty} {p(n)} = p(0) + p(1) + p(2) + p(3) + p(4) = 1$$
    $$p(0) + p(1) + p(2) + p(3) = 1$$
    $$p(0) + 1.720 p(0) + 0.692 p(0) + 0.462 p(0) + 0.231 p(0) = 1$$
    $$4.105 p(0) = 1$$
    $$p(0) = \frac{1}{4.105}$$
    $$p(0) = 0.244$$

  $$p(1) = 0.420$$
  $$p(2) = 0.169$$
  $$p(3) = 0.113$$
  $$p(4) = 0.056$$

Por lo tanto:
$$\overline{L} = 0 * 0.244 + 1 * 0.420 + 2 * 0.169 + 3 * 0.113 + 4 * 0.056 = 1.321 $$

$$J = 4 - 1.321 = 2.679 $$
  
Calculando el beneficio obtenido con este nuevo modelo se observa que:
$Beneficio = 2.679 * 50\$ * 24 dias =  3214.8\$ $

Con lo cual, se puede concluir que como el beneficio obtenido se incrementa en 403.2\$ por mes y contratar un nuevo empleado tiene un costo de 700\$ no es conveniente contratar a esta nueva persona.



\paragraph{Ejercicio 6}
    Se puede modelar la peluquería como un P/P/M con M=2, ya que ambos peluqueros atienden en paralelo. \\
    Las hipótesis del modelos son:
    \begin{itemize}
     \item El arribo de clientes y el proceso de servicio corresponden a una distribución Poisson.
     \item Hay 2 canales, de los cuales uno atiende si al entrar un cliente, hay una persona esperando.
     \item Se analiza el sistema en Régimen Permanente o Estacionario.
     \item La impaciencia está dada por la tabla que se especifica.
     \item La capacidad del sistema es infinita.
     \item La población es infinita.
     \item Hay una única cola frente al canal.
    \end{itemize}

    Si bien el sistema es ilimitado, los clientes son impacientes con lo cual hay un número finito de estados posibles. La función de impaciencia es:
    \begin{center}
    \begin{tabular}{|| c | c ||}
    \hline 
    n & PI(n) \\ \hline \hline
    0 & 1\\ \hline
    1 & 1\\ \hline
    2 & 0,5\\ \hline
    3 & 0,5\\ \hline
    4 & 0\\ \hline
    
    \end{tabular}
    \end{center}
    La siguiente tabla describe al sistema:
    \begin{center}
    \begin{tabular}{|| c | c | c | c | c | c | c | c ||}
    \hline 
     n & P(n) & $\lambda_n$ & $\mu_n$ & $L_n$& $L_cn$ & $H_n$ & $R_n$ \\ \hline \hline
     0 & P(0) & $\lambda$   & 0       & 0    & 0      & 0     & 0	\\ \hline
     1 & P(1) & $\lambda$   & $\mu$   & 1    & 0      & 1     & 0	\\ \hline
     2 & P(2) & $\lambda$/2 & $\mu$   & 2    & 1      & 1     & $\lambda$/2	\\ \hline
     3 & P(3) & $\lambda$/2 & 2$\mu$  & 3    & 1      & 2     & $\lambda$/2 \\ \hline
     4 & P(4) & 0           & 2$\mu$  & 4    & 2      & 2     & $\lambda$   \\ \hline  
     
    \end{tabular}
    \end{center}
    \underline{Datos:} \\
    $$t_s = 0,5 h/cl  \Rightarrow \mu = 2 cl/h $$
    $$\lambda = 5 cl/h $$
    
    \paragraph{6.a.} La probabilidad de que no haya clientes en la peluquería es P(0). Para eso, se calcurarán todas las probabilidades a partir de la tabla,
    resolviendo el siguiente sistema de 5 ecuaciones y 5 incognitas.\\
    Se calcula la probabilidad de un estado mediante la fórmula $p(n) = \frac{\lambda_(n-1)}{\mu_n} p(n-1)$
    $$ \sum_{n=0}^{4} p(n) = 1 $$
    $$ p(1) = \frac{\lambda_0}{\mu_1}p(0) = \frac{\lambda}{\mu}p(0) = \frac{5}{2}p(0)$$
    $$ p(2) = \frac{\lambda_1}{\mu_2}p(1) = \frac{\lambda/2}{\mu}p(1) = \frac{5}{4}p(1) = \frac{25}{8}p(0)$$
    $$ p(3) = \frac{\lambda_2}{\mu_3}p(2) = \frac{\lambda/2}{2\mu}p(2) = \frac{5}{8}p(2) = \frac{125}{64}p(0)$$
    $$ p(4) = \frac{\lambda_3}{\mu_2}p(3) = \frac{\lambda/2}{2\mu}p(3) = \frac{5}{8}p(3) = \frac{625}{512}p(0)$$
    
    Despejando $p(0)$ del sistema y reemplazando los valores se obtienen las siguientes probabilidades:
    $$ p(0) = \frac{512}{5017} \simeq 0,102$$
    $$ p(1) = \frac{1280}{5017} \simeq 0,255$$
    $$ p(2) = \frac{1600}{5017} \simeq 0,319$$
    $$ p(3) = \frac{1000}{5017}\simeq 0,199$$
    $$ p(4) = \frac{625}{5017} \simeq 0,125$$
    
    Por lo tanto, la probabilidad de que no haya clientes en la peluquería es $p(0) = 0,102$
       
    
    \paragraph{6.b.} La probabilidad de que haya clientes esperando se relaciona con la probabilidad de los estados 2, 3 y 4. \\
    $$P(2) + P(3) + P(4) = \frac{3225}{5017} \simeq 0,643$$
    
    \paragraph{6.c.} El porcentaje de ocupación del peluquero es $H_p$. Para eso podemos calcular H por definicion:
    $$H_p = \sum_{n=0}^{4}H_p(n) p(n) = 0 p(0) + 1 p(1) + 1 p(2) + 1 p(3) + 1 p(4) \simeq 0,898 \simeq 90\% $$
    Para el aprendiz:
    $$H_a = \sum_{n=0}^{4}H_a(n) p(n) = 0 p(0) + 0 p(1) + 0 p(2) + 1 p(3) + 1 p(4) \simeq 0,324 \simeq 32\% $$
    
    \paragraph{6.d.} La cantidad promedio de clientes esperando se obtiene mediante $L_c$.
    $$L_c = \sum_{n=0}^{4}L_c(n) p(n) =1 p(2) + 1 p(3) + 2 p(4) = \frac{3850}{5017} \simeq 0,767$$
    
    
    \paragraph{6.e.} La cantidad promedio de clientes que no ingresan es igual al rechazo  $ \overline{R}$
    En este caso se puede calcular como $ \overline{R} = \lambda - \overline{\lambda}$ (Se tomará el valor de $\overline{\lambda}$ del punto siguiente.)
    $$ \Rightarrow \overline{R} = 5 - 3,08 \simeq 1,92 \frac{clientes}{hora}$$
    
    
    \paragraph{6.f.} Para obtener el ingreso por hora, se debe calcular la cantidad de clientes atendidos por hora, es decir $\overline{\mu}$.\\
    Se sabe que la cantidad de clientes atendidos es igual a la cantidad de clientes que efectivamente ingresan, entonces \\
    $$ \overline{\lambda} = \sum_{n=0}^{4}\lambda(n) p(n) = \lambda [ p(0)+p(1)] + \frac{\lambda}{2} [p(2) + p(3)] = \frac{15460}{5017} \simeq 3.08 \frac{clientes}{hora}$$
 

\paragraph{Ejercicio 7}
  El sistema planteado se puede dividir en 3 subsistemas A,B y C como indica el gráfico del enunciado. 
  El sistema A es un sistema P/P/1 impaciente, B es P/P/1 y C es un sistema P/P/1/2.
  
  Se cuenta con los siguientes datos: \\
  $$ \lambda = 10 cl/min $$
  $$ \mu_a = 1/t_a = 5 cl/min $$
  $$ \mu_b = 1/t_b = 3 cl/min $$
  $$ \mu_c = 1/t_c = 2 cl/min $$
 
  Por las características del sistema, se deducen las siguientes igualdades: \\
  $$ \overline{\lambda_a} = \overline{\mu_a}$$
  $$ \lambda_b = \overline{\lambda_b} = 0.4 \overline{\lambda_a} = \overline{\mu_b}$$
  $$ \lambda_c = 0.6 \overline{\lambda_a} = \overline{R_c} + \overline{\lambda_c} , \overline{\lambda_c} = \overline{\mu_c}$$
  
 Para los sistemas A y C se planteará una tabla y para el sistema B se utilizarán las fórmulas de PP1. \\ \\

 \begin{center}
  \underline{Sistema A} \\ 
    \begin{tabular}{|| c | c | c | c | c | c | c | c ||}
    \hline 
     n & P(n) & $\lambda_n$ & $\mu_n$ & $L_n$& $L_cn$ & $H_n$ & $R_n$ 			\\ \hline \hline
     0 & P(0) & $\lambda_a$   & 0       & 0    & 0      & 0     & 0			\\ \hline
     1 & P(1) & $\lambda_a$/2 & $\mu_a$ & 1    & 0      & 1     & $\lambda_a$/2		\\ \hline
     2 & P(2) & $\lambda_a$/5 & $\mu_a$ & 2    & 1      & 1     & $\frac{4\lambda_a}{5}$	\\ \hline
     3 & P(3) &  0            & $\mu_a$ & 3    & 2      & 1     & $\lambda_a$ 		\\ \hline
     
    \end{tabular}
    
    Las probabilidades de cada estado se resuelven mediante el sistema: 
    $$ \sum_{n=0}^{3}p(n) = 1 $$
    $$ p(1)_a = \frac{\lambda_a}{\mu_a} p(0)_a = \frac{10}{5} p(0)_a $$
    $$ p(2)_a = \frac{\lambda_a}{2\mu_a} p(1)_a = \frac{10}{5} p(0)_a$$
    $$ p(3)_a = \frac{\lambda_a}{5\mu_a} p(2)_a = \frac{4}{5} p(0)_a$$
    
    Despejando se obtiene: 
    $$ p(0)_a = \frac{5}{29} \simeq 0,17 $$
    $$ p(1)_a = \frac{10}{29} \simeq 0,345 $$
    $$ p(2)_a = \frac{10}{29} \simeq 0,345 $$
    $$ p(3)_a = \frac{4}{29} \simeq 0,14 $$
    
  \end{center} 
  
 \begin{center}
  \underline{Sistema C} \\ 
    \begin{tabular}{|| c | c | c | c | c | c | c | c ||}
    \hline 
     n & P(n) & $\lambda_n$ & $\mu_n$ & $L_n$& $L_cn$ & $H_n$ & $R_n$ 	\\ \hline \hline
     0 & P(0) & $0.6 \overline{\lambda_a}$ & 0       & 0    & 0      & 0     & 0	\\ \hline
     1 & P(1) & $0.6 \overline{\lambda_a}$ & $\mu_c$ & 1    & 0      & 1     & 0	\\ \hline
     2 & P(2) & 0               & $\mu_c$ & 2    & 1      & 1     & $0.6 \overline{\lambda_a}$ \\ \hline
     
    \end{tabular}
    
    Las probabilidades de cada estado se resuelven mediante el sistema: 
    $$ \sum_{n=0}^{3}p(n) = 1 $$
    $$ p(1)_c = \frac{0.6\overline{\lambda_a}}{\mu_c} p(0)_c$$
    $$ p(2)_c = \frac{0.6\overline{\lambda_a}}{2\mu_c} p(1)_c$$
    
    Para esto, es necesario obtener $\overline{\lambda_a}$: \\
    $$\overline{\lambda_a} = \sum_{n=0}^{3}\lambda(n) p(n)_a = \lambda p(0)_a + \frac{\lambda}{2} p(1)_a + \frac{\lambda}{5} p(2)_a = \frac{50}{29} + \frac{50}{29} + \frac{20}{29} = \frac{120}{29} \simeq 4.14$$
    
    Reemplazando se obtiene obtiene: 
    $$ p(1)_c = \frac{3}{5}\frac{120}{29}\frac{1}{2} p(0) = \frac{36}{29} p(0) $$
    $$ p(2)_c = \frac{3}{5}\frac{120}{29}\frac{1}{2} p(1) = \frac{1296}{841} p(0) $$
    
    Por último, despejando p(0) y reemplazando en las ecuaciones:
    $$ p(0)_c = \frac{841}{3181} \simeq 0,264 $$
    $$ p(1)_c = \frac{1044}{3181} \simeq 0,328 $$
    $$ p(2)_c = \frac{1296}{3181} \simeq 0,407 $$
  \end{center}
  
  \paragraph{7.a.} La cantidad promedio de clientes esperando en cada sector se obtiene mediante $\overline{L_ca}, \overline{L_cb},\overline{L_cc}$, para los sectores A,B y C respectivamente.\\
  Para los sistemas A y C se puede calcular por definición:
  $$\overline{L_ca} = \sum_{n=0}^{3}L_ca(n) p(n)_a = 1 p(2)_a + 2 p(3)_a = \frac{14}{29} \simeq 0,48 $$
  $$\overline{L_cc} = \sum_{n=0}^{2}L_cc(n) p(n)_c = p(2)_c = \frac{1296}{3181} \simeq 0,407  $$
  
  Para el sistema B se utiliza la fórmula para PP1 $\overline{L_c} = \frac{\rho^2}{1-\rho}$, con $\rho = \frac{0.4\overline{\lambda_a}}{\mu_b}$ utilizando las igualdades mencionadas anteriormente. Ya tenemos el $\overline{\lambda_a}$ calculado con lo cual \\
  $$\Rightarrow \rho = \frac{0.4\frac{120}{29}}{3} = \frac{16}{29} $$
  $$\Rightarrow \overline{L_cb} = \frac{(\frac{16}{29})^2}{1-\frac{16}{29}} = \frac{256}{377} \simeq 0,68 $$
  
  \paragraph{7.b.} El ingreso promedio por minuto se calcula como $\$5000 \overline{\mu_b} + \$600 \overline{\mu_c} $. Considerando las igualdades mencionadas, se puede deducir:
  $$ \overline{\mu_b} = 0.4\overline{\lambda_a} = \frac{48}{29} \simeq 1,66 \frac{clientes}{minuto}$$ 
  
  Para $\overline{\mu_c} = \overline{\lambda_c}$ se debe calcular primero el rechazo en C
  $$ \overline{R_c} = \sum_{n=0}^{2} R_c(n) p_c(n) = 0.6 \overline{\lambda_a} p_c(2) =  \frac{93312}{92249} \simeq 1,012 \frac{clientes}{minuto}$$
  $$\Rightarrow \overline{\lambda_c} = \lambda_c -\overline{R_c} = 0.6\overline{\lambda_a} -\overline{R_c} \simeq 2,483 - 1,012 = 1,471 \frac{clientes}{minuto}$$
  
  Con lo cual, el ingreso se calcula como
  $$\$5000 \overline{\mu_b} + \$600 \overline{\mu_c} = \$5000 \frac{48}{29} + \$600 1,471 = \$8.275,86 + \$882,6 = \$9.158,46 / minuto$$
  
  \paragraph{7.c.} La cantidad promedio de clientes que no ingresan al sistema por minuto se calcula mediante $\overline{R_a}$. 
  \underline{Hipótesis:} se consideran sólo los clientes que son rechazados en A dado que los que son rechazados en C ya recibieron parte del servicio.
  $$\overline{R_a} = \lambda_a -\overline{\lambda_a} = 10 - \frac{120}{29} = \frac{170}{29} \simeq 5,86 \frac{clientes}{minuto}$$
  
  \paragraph{7.d.} La probabilidad de que el sistema esté vacío se calcula mediante las probabilidades de que cada sistema este vacío en simultáneo. 
  Para B se utiliza la fórmula de PP1 donde $p(0) = 1-\rho = 1- \frac{16}{29} = \frac{13}{29} $
  $$ p(0)_a . p(0)_b . p(0)_c = \frac{5}{29} .\frac{13}{29} . \frac{841}{3181} = \frac{65}{3181} \simeq 0,020 $$

\paragraph{Ejercicio 8}
   Para modelar el sistema presentado en este ejercicio, se plantean las siguientes hipótesis:

   \begin{itemize}
      \item Los procesos de arribo y atención de clientes son de tipo Poisson.
      \item Se cuenta con un único canal de atención.
      \item La capacidad del sistema es infinita.
      \item La disciplina de atención es FIFO.
      \item El sistema se encuentra en régimen estacionario.
      \item El sistema tiene sólo una cola.
      \item La población de clientes es infinita.
      \item No hay impaciencia.
   \end{itemize}

   \underline{Datos:} \\
   $$\lambda = 8 cl/min $$
   $$\mu = 10 cl/min $$

   \paragraph{Cálculo de $\overline{\lambda}$}


   Debido a que el sistema es un PP1 con reciclaje:
      $$\overline{\lambda} = \lambda + 0.1 \overline{\mu}$$
   
   Y, como además, el sistema se encuentra en equilibrio:
      $$\overline{\lambda} = \overline{\mu}$$
   
   Por lo tanto:
      $$ \overline{\lambda} = \lambda + 0.1 \overline{\lambda} $$
      $$ \overline{\lambda} - 0.1 \overline{\lambda} = \lambda $$
      $$ \overline{\lambda} = \lambda / 0.9 \simeq 8.889cl/min $$

   \paragraph{8.a.} La cantidad promedio de clientes en la cola es $ \overline{L_{c}}$.
      $$ \overline{L_{c}} = \frac{\overline{\lambda}^2}{\mu (\mu - \overline{\lambda})}$$
      $$ \overline{L_{c}} \simeq  \frac{(8.889cl/min)^2}{10cl/min \cdot (10cl/min - 8.889cl/min)} $$
      $$ \overline{L_{c}} \simeq  7.111 $$
   
      Por lo tanto, la cantidad promedio es de $7.111$ clientes en la cola.


   \paragraph{8.b.} El tiempo promedio de permanencia de un cliente en la cola es $W_c$.
      $$ W_c = \frac{L_c}{\overline{\lambda}} $$
      $$ W_c \simeq 7.111cl / 8.889cl/min $$
      $$ W_c \simeq 0.8min $$

      Por lo tanto, el tiempo de permanencia en la cola, en promedio, es de $0.8$ minutos.


   \paragraph{8.c.}
      $$ p(canal\:no\:ocioso) = p(n \ge 1) = 1 - p(0) $$
      $$ ... = \rho $$
      $$ ... = \frac{\overline{\lambda}}{\mu} $$
      $$ ... \simeq \frac{8.889cl/min}{10cl/min} $$
      $$ \Rightarrow p(canal\:no\:ocioso) \simeq 0.889 $$

   \paragraph{8.d.} La cantidad de clientes promedio en el sistema es $L$.
      $$ L = \frac{\overline{\lambda}}{\mu - \overline{\lambda}}$$
      $$ L \simeq \frac{8.889cl/min}{10cl/min - 8.889cl/min} $$
      $$ L \simeq 8 $$


\paragraph{Ejercicio 9}
   Para modelar el sistema presentado en este ejercicio, se plantean las siguientes hipótesis:

   \begin{itemize}
      \item Población infinita.
      \item No hay impaciencia.
      \item Clientes sin preferencia entre ventanillas 1 y 2 o entre 3 y 4.
      \item En caso de $(b, b, 1, 1)$, al terminar C3 o C4, tiene prioridad el cliente de C1.
   \end{itemize}

   \paragraph{9.a.} Los estados posibles del sistema son:
   \begin{center}
   \begin{tabular}{|| c | c | c | c | c ||}
   \hline 
      n & C1 & C2 & C3 & C4 \\ \hline \hline
      0 & 0  & 0  & 0  & 0  \\ \hline \hline
      1 & 1  & 0  & 0  & 0  \\ \hline 
      1 & 0  & 1  & 0  & 0  \\ \hline
      1 & 0  & 0  & 1  & 0  \\ \hline
      1 & 0  & 0  & 0  & 1  \\ \hline \hline
      2 & 1  & 1  & 0  & 0  \\ \hline
      2 & 1  & 0  & 1  & 0  \\ \hline
      2 & 1  & 0  & 0  & 1  \\ \hline
      2 & 0  & 1  & 1  & 0  \\ \hline
      2 & 0  & 1  & 0  & 1  \\ \hline
      2 & 0  & 0  & 1  & 1  \\ \hline \hline
      3 & 1  & 1  & 1  & 0  \\ \hline
      3 & 1  & 1  & 0  & 1  \\ \hline
      3 & 1  & 0  & 1  & 1  \\ \hline
      3 & b  & 0  & 1  & 1  \\ \hline
      3 & 0  & 1  & 1  & 1  \\ \hline
      3 & 0  & b  & 1  & 1  \\ \hline \hline
      4 & 1  & 1  & 1  & 1  \\ \hline
      4 & 1  & b  & 1  & 1  \\ \hline
      4 & b  & 1  & 1  & 1  \\ \hline
      4 & b  & b  & 1  & 1  \\ \hline
   \end{tabular}
   \end{center}

   \paragraph{9.b.}
      $$p(b,0,1,1) = p(1,0,1,1) (1 - \lambda \Delta t) (\mu_1 \Delta t) (1 - \mu_3 \Delta t) (1 - \mu_4 \Delta t)$$
      $$           + p(b,0,1,1) (1 - \lambda \Delta t) (1 - \mu_3 \Delta t) (1 - \mu_4 \Delta t)$$
      $$           + p(b,b,1,1) 0.5 (\mu_3 \Delta t) (1 - \mu_4 \Delta t) $$
      $$           + p(b,b,1,1) 0.5 (1 - \mu_3 \Delta t) (\mu_4 \Delta t) $$

   \paragraph{9.c.}
      $$p(0,0,1,1) = p(1,0,1,0) (1 - \lambda \Delta t) (\mu_1 \Delta t) (1 - \mu_3 \Delta t)$$
      $$           + p(1,0,0,1) (1 - \lambda \Delta t) (\mu_1 \Delta t) (1 - \mu_4 \Delta t)$$
      $$           + p(0,1,1,0) (1 - \lambda \Delta t) (\mu_2 \Delta t) (1 - \mu_3 \Delta t)$$
      $$           + p(0,1,0,1) (1 - \lambda \Delta t) (\mu_2 \Delta t) (1 - \mu_4 \Delta t)$$
      $$           + p(0,0,1,1) (1 - \lambda \Delta t) (1 - \mu_3 \Delta t) (1 - \mu_4 \Delta t)$$
      $$           + p(b,0,1,1) (1 - \lambda \Delta t) [(\mu_3 \Delta t) (1 - \mu_4 \Delta t) + (1 - \mu_3 \Delta t) (\mu_4 \Delta t)]$$
      $$           + p(0,b,1,1) (1 - \lambda \Delta t) [(\mu_3 \Delta t) (1 - \mu_4 \Delta t) + (1 - \mu_3 \Delta t) (\mu_4 \Delta t)]$$


\paragraph{Ejercicio 10}
   Para modelar el sistema presentado en este ejercicio, se plantean las siguientes hip\'otesis:

   \begin{itemize}
      \item Poblaci\'on infinita.
      \item No hay impaciencia. No se admite abandono
   \end{itemize}

   \paragraph{10.a.} Los estados posibles del sistema son:
   \begin{center}
   \begin{tabular}{|| c | c | c | c | c ||}
   \hline 
      n & C1 & C2 & C3 \\ \hline \hline
      0 & 0  & 0  & 0   \\ \hline \hline
      1 & 1  & 0  & 0    \\ \hline 
      1 & 0  & 1  & 0    \\ \hline
      1 & 0  & 0  & 1    \\ \hline \hline
      2 & 1  & 1  & 0   \\ \hline
      2 & b  & 1  & 0    \\ \hline
      2 & 1  & 0  & 1    \\ \hline
      2 & b  & 0  & 1    \\ \hline
      2 & 0  & 1  & 1    \\ \hline
      2 & 0  & b  & 1    \\ \hline \hline
      3 & 1  & 1  & 1    \\ \hline
      3 & b  & 1  & 1    \\ \hline
      3 & 1  & b  & 1    \\ \hline
      3 & b  & b  & 1    \\ \hline
   \end{tabular}
   \end{center}


   \paragraph{10.b.}
      $$p(0,1,1) = p(1,1,0) 0.2 (\mu_1 \Delta t) (1 - \mu_2 \Delta t) $$
      $$           + p(b,1,0) (\mu_2 \Delta t) $$
      $$           + p(1,0,1) 0.8 (\mu_1 \Delta t) (1 - \mu_3 \Delta t) $$
      $$           + p(0,1,1) (1 - \lambda \Delta t) (1 - \mu_2 \Delta t) (1 - \mu_3 \Delta t) $$
      $$           + p(b,1,1) 0.2 (1 - \mu_2 \Delta t) (\mu_3 \Delta t) $$
      $$           + p(b,b,1) 0.8 (\mu_3 \Delta t) $$

   \paragraph{10.c.}
      $$p(1,0,1) = p(0,0,1) (\lambda \Delta t) (1 - \mu_3 \Delta t) $$
      $$           + p(1,1,0) (1 - \mu_1 \Delta t)(\mu_2 \Delta t) $$
      $$           + p(1,0,1) (1 - \mu_1 \Delta t) (1 - \mu_3 \Delta t) $$
      $$           + p(1,b,1) (1 - \mu_1 \Delta t) (\mu_3 \Delta t) $$


\end{document}
