\documentclass{article}

\usepackage[spanish]{babel}
\selectlanguage{spanish}
\usepackage[utf8]{inputenc}

\def \anio {a\tilde{n}o}

\title{Trabajo Práctico 2\\Administración de Stocks}
\author{}
\date{\today}

%%% Agregar integrantes del grupo %%%

\begin{document}
\maketitle

%%%% Poner el enunciado aca.. y un indice %%%%
\paragraph{Ejercicio 4}
    Los datos proporcionados son la demanda anual $ D = 20000 u $, el costo de setup $ K = \$6000 $, el costo de almacenamiento $ C_1 = \$20\ u^{-1}\ \anio^{-1} $ y la tasa de producción $ p = 5000u\ mes^{-1} = 60000u\ \anio^{-1}$.

    \paragraph{4.a}
    El modelo elegido para este problema es el Modelo de Reposición no Instantánea. Se asumen las siguientes hipótesis:
        \begin{itemize}
            \item Se administra un solo producto.
            \item La demanda es conocida y constante.
            \item No hay descuentos por cantidad.
            \item No hay inflación.
            \item La producción se efectúa a tasa conocida y constante.
            \item No se admite agotamiento.
            \item No hay stock de protección.
            \item Costo de setup independiente del tamaño del lote.
            \item Costo unitario de almacenimiento independiente del stock.
            \item Se supone continuidad permanente de operación.
        \end{itemize}

    \paragraph{4.b.} 
        $$ q_o = \sqrt{ \frac{2KD}{TC_1 \left( 1 - \frac{d}{p} \right)} } $$
        $$ q_o = \sqrt{ \frac{2 \cdot \$ 6000 \cdot 20000u\ \anio^{-1}}{\$20 u^{-1}\ \anio^{-1} \left( 1 - \frac{20000u\ \anio^{-1}}{60000u\ \anio^{-1}} \right)} } $$
        $$ q_o = \sqrt{ \frac{2 \cdot 6 \cdot 10^6\ u^2}{1 - \frac{2}{6}} } = q_o = \sqrt{ \frac{2 \cdot 6 \cdot 10^6}{\frac{2}{3}} }\ u = \sqrt{ 3 \cdot 6 \cdot 10^6 }\ u $$
        $$ q_o = 3 \cdot 10^3 \sqrt {2}\ u \simeq 4242.64\ u $$

    \paragraph{4.c.}
        Como se indica en el item h se consideran 20 días por mes.
        
        $$ \frac{T}{t_i} = \frac{D}{q} \Rightarrow t_i = \frac{Tq}{D} $$
        $$ t_i = \frac{3 \cdot 10^3 \sqrt{2}\ u}{20000u\ \anio^{-1}} = \frac{3 \sqrt{2}\ \anio}{20} \frac{240 dias}{\anio} $$
        $$ t_i = 36 \sqrt{2} dia \simeq 50.91\ dias $$

\end{document}
